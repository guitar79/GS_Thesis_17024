\maketitle  % command to print the title page with above variables
\setcounter{page}{1}
%---------------------------------------------------------------------
%                  영문 초록을 입력하시오
%---------------------------------------------------------------------
\begin{abstracts}     %this creates the heading for the abstract page
	\addcontentsline{toc}{section}{Abstract}  %%% TOC에 표시
	\noindent{
		Stars are born when matter from interstellar molecular clouds fall to the center to increase the mass of the protostar. Bipolar outflows are formed to remove the excess angular momentum of falling matter. Intensities of outflows are known to be in a close relationship with their bolometric luminosity and evolutionary stages. In this study, data from Institute for Radio Astronomy in the Millimeter Range (IRAM) 30$\,$m Telescope and Taeduk Radio Astronomy Observatory (TRAO) were used. IRAM data were used to map $^{12}\textrm{CO}$ J = 2 - 1 over Orion A molecular cloud. TRAO data were used to map $^{13}\textrm{CO}$ J = 1 - 0 over the same region. Outflows were observed and measured by drawing contour maps and line profiles of  red/blue shifted components. The correlation between a protostar's luminosity and outflow momentum flux have been confirmed. Also, outflows could be detected better if the energy level of the emission line is higher. 
	}
\end{abstracts}

\begin{abstractskor}
	Perovskite는 태양 전지, LED등의 여러 광전소자 분야에서 기존의 것들에 비해 더 좋은 성능과 값싼 가격, 쉬운 제조 방법으로 인해 각광받고 있는 물질이다. 대표적인 페로브 스카이트 물질인 CsPbBr3 단결정에 레이저를 쏘았을 때에 결정의 바깥쪽에서 빛이 나오는 현상을 보고 wave guiding effect의 가능성이 있다고 판단하였다. 본 연구는 그것의 원인이 무엇인지 탐구하고 원인 분석을 통하여 그것의 발전 가능성과 방향 제시를 한다. 
	 그 방법은 단결정을 PL로 찍어서 나오는 peak들을 분석하는 것이며, exciton과 biexciton의 peak을 Origin 프로그램을 통하여 분석 할 수 있다.
\end{abstractskor}
%----------------------------------------------
%   Table of Contents (자동 작성됨)
%----------------------------------------------
\cleardoublepage
\addcontentsline{toc}{section}{Contents}
\setcounter{secnumdepth}{3} % organisational level that receives a numbers
\setcounter{tocdepth}{3}    % print table of contents for level 3
\baselineskip=2.2em
\tableofcontents


%----------------------------------------------
%     List of Figures/Tables (자동 작성됨)
%----------------------------------------------
\cleardoublepage
\clearpage
\listoffigures	% 그림 목록과 캡션을 출력한다. 만약 논문에 그림이 없다면 이 줄의 맨 앞에 %기호를 넣어서 코멘트 처리한다.

\cleardoublepage
\clearpage
\listoftables  % 표 목록과 캡션을 출력한다. 만약 논문에 표가 없다면 이 줄의 맨 앞에 %기호를 넣어서 코멘트 처리한다.

%%%%%%%%%%%%%%%%%%%%%%%%%%%%%%%%%%%%%%%%%%%%%%%%%%%%%%%%%%%
%%%% Main Document %%%%%%%%%%%%%%%%%%%%%%%%%%%%%%%%%%%%%%%%
%%%%%%%%%%%%%%%%%%%%%%%%%%%%%%%%%%%%%%%%%%%%%%%%%%%%%%%%%%%
\cleardoublepage
\clearpage
\renewcommand{\thepage}{\arabic{page}}
\setcounter{page}{1}



