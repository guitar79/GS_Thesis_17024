\maketitle  % command to print the title page with above variables
\setcounter{page}{1}
%---------------------------------------------------------------------
%                  영문 초록을 입력하시오
%---------------------------------------------------------------------
\begin{abstracts}     %this creates the heading for the abstract page
	\addcontentsline{toc}{section}{Abstract}  %%% TOC에 표시
	\noindent{
		Perovskite is attracting attention due to its superior performance, cheap price and easy manufacturing method compared to conventional ones in various photoelectric devices such as solar cell and LED. It is considered that there is a possibility of wave guiding effect due to the phenomenon that light is emitted from the outside of the crystal when a laser is fired in a typical perovskite material $\rm CsPbBr_3$ single crystal.
		
		This study analyzes the PL data using the NT-MDT instrument to determine what causes the wave guiding effect and the process of it. The measured PL data are analyzed by using the Origin 9.0 program. From the analyzed data, we observe the tendency of PL data from the center to the edge of the single crystal.
	}
\end{abstracts}

\begin{abstractskor}
	페로브스카이트는 태양 전지, LED등의 여러 광전소자 분야에서 기존의 것들에 비해 더 좋은 성능과 값싼 가격, 쉬운 제조 방법으로 인해 각광받고 있는 물질이다. 대표적인 페로브스카이트 물질인 $\rm CsPbBr_3$ 단결정에 레이저를 쏘았을 때에 결정의 바깥쪽에서 빛이 나오는 현상을 보고 그 원인으로 wave guiding effect의 가능성이 있다고 판단하였다. 
	
	본 연구는 어떠한 것이 wave guiding effect를 일으키고 그 과정을 알아내기 위해서 NT-MDT 기기를 이용하여 PL data를 분석한다. 측정한 PL data는 Origin 9.0 프로그램을 이용하여 분석하며, 단결정의 내부에서 가장자리로 가면서의 변화를 관찰한다.
\end{abstractskor}
%----------------------------------------------
%   Table of Contents (자동 작성됨)
%----------------------------------------------
\cleardoublepage
\addcontentsline{toc}{section}{Contents}
\setcounter{secnumdepth}{3} % organisational level that receives a numbers
\setcounter{tocdepth}{3}    % print table of contents for level 3
\baselineskip=2.2em
\tableofcontents


%----------------------------------------------
%     List of Figures/Tables (자동 작성됨)
%----------------------------------------------
\cleardoublepage
\clearpage
\listoffigures	% 그림 목록과 캡션을 출력한다. 만약 논문에 그림이 없다면 이 줄의 맨 앞에 %기호를 넣어서 코멘트 처리한다.

\cleardoublepage
\clearpage
\listoftables  % 표 목록과 캡션을 출력한다. 만약 논문에 표가 없다면 이 줄의 맨 앞에 %기호를 넣어서 코멘트 처리한다.

%%%%%%%%%%%%%%%%%%%%%%%%%%%%%%%%%%%%%%%%%%%%%%%%%%%%%%%%%%%
%%%% Main Document %%%%%%%%%%%%%%%%%%%%%%%%%%%%%%%%%%%%%%%%
%%%%%%%%%%%%%%%%%%%%%%%%%%%%%%%%%%%%%%%%%%%%%%%%%%%%%%%%%%%
\cleardoublepage
\clearpage
\renewcommand{\thepage}{\arabic{page}}
\setcounter{page}{1}



