%-----------------------------------------------------
% Conclusion
%-----------------------------------------------------
\newpage

\section{결론 및 고찰}
본 연구의 주요 결론은 다음과 같다:
\begin{enumerate}
	\item Perovskite $CsPbBr_3$의 single crystal은 결정의 바깥 쪽에서 wave guiding effect가 일어난다.
	\item 결정의 가장자리에서 exciton 과 biexciton의 경향성이 반대가 되며, 종합적으로는 중앙에서 가장자리로 가면서 감소했다가 다시 증가하는 추세를 가지고 있다. 
\end{enumerate}
같은 ND2로 찍은 PL 데이터를 관찰했을 때, 완전한 가장자리를 제외하면 바깥으로 갈 수록 biexciton peak의 상대적인 세기가 세짐을 관찰할 수 있었다. 
변하지 않는 구조를 갖는 CsPbBr3에 동일한 레이저를 가하기 때문에 비슷한 양의 carrier가 전도띠로 가는 것은 자명하다. 이 carrier들은 각각 exciton이나 biexciton의 형태로 존재하게 되는데, PL에서 biexciton에 의해 형성되는 shoulder peak가 더 우세하게 관찰된 것이라고 해석할 수 있다. 

PL 데이터로 fitting 한 peak의 intensity가 커진다는 것은 exciton과 biexciton이 생성되는 radiative recombination이 많아진다는 것을 의미하고 이는 defect가 줄어 결정의 순도가 높아지는 것으로 해석할 수 있다.

반대로 생각해보면 중심으로 갈수록 증가하는 PL peak는 결정의 가장자리 부분으로 갈수록 결정의 수도가 낮다고 판단될 만큼의 defect가 존재했다는 것을 의미한다. 하지만 실험 결과를 보면 완전한 가장자리에서는 다시 증가하는 것을 볼 수 있는데, 이는 waveguiding effect에 의한 것으로 보인다. 

Wave guiding effect 의 원인은 다음과 같이 설명할 수 있다: defect의 에너지 준위는 전도띠와 원자가띠 사이에 존재하므로, 전도띠에 있는 전자는 가까운 defect의 에너지 준위로 내려가기를 선호한다. 즉, defect가 많은 가장자리로 전자가 모이게 된다. 실험 결과의 분석을 통해 알아낼 수 있었다. 