%-----------------------------------------------------
% Conclusion
%-----------------------------------------------------
\newpage

\section{결론	}
본 연구의 주요 결론은 다음과 같다:
\begin{enumerate}
	\item Perovskite $\rm CsPbBr_3$의 single crystal은 PL data 의 경향성을 보았을 때 결정의 바깥 쪽에서 wave guiding effect가 발생하고 있다고 해석할 수 있다.
	\item 결정의 가장자리에서 exciton 과 biexciton의 경향성이 반대가 되며, 종합적으로는 중앙에서 가장자리로 가면서 감소했다가 다시 증가하는 추세를 가지고 있다. 이는 결국 defect가 많은 가장자리로 전자가 많이 모이게 됨을 뜻한다.
\end{enumerate}
PL 데이터로 fitting 한 peak의 intensity가 커진다는 것은 exciton과 biexciton이 생성되는 radiative recombination이 많아진다는 것을 의미하고 이는 defect가 줄어 결정의 순도가 높아지는 것으로 해석할 수 있다.

반대로 생각해보면 중심으로 갈수록 증가하는 PL peak는 결정의 가장자리 부분으로 갈수록 결정의 순도가 낮다고 판단될 만큼의 defect가 존재했다는 것을 의미한다. 하지만 실험 결과를 보면 완전한 가장자리에서는 다시 증가하는 것을 볼 수 있는데, 이는 waveguiding effect에 의한 것으로 보인다. 이론적 배경에서도 언급했듯이 waveguiding effect는 빛이 에너지의 이득을 보기 위해서 특정 장소로 모이게 되는 일이 발생하게 되는 것을 말하는데, 결과를 보면 본 논문도 같은 경우로 보인다.

Defect의 에너지 준위는 전도띠와 원자가띠 사이에 존재하므로, 전도띠에 있는 전자는 가까운 defect의 에너지 준위로 내려가기를 선호한다. 실험 결과를 보았을 때 가장자리로 갈수록 defect가 많았으리라 추정할 수 있다. Defect가 많으면 그 에너지 준위로 전자가 많이 이동하기 때문에 이것이 가장자리로 에너지가 모이는 waveguiding effect를 발생시킨다고 볼 수 있다.