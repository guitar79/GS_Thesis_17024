\section{연구 과정 및 결과}

\subsection{샘플 제작}
기존의 페로브스카이트 결정을 만드는 방법과는 다르게 간단하고 빠른 PDMS stamping 방법을 사용하였다.
모든 용액은 실온과 공기 중에서 제작되었다. CsPbBr3의 용액을 만들기 위해서 CsBr과 PbBr2를 1:1의 몰 비율로 섞었으며 용매는 Dimethyl Sulfoxide(DMSO)를 이용하였다. 용매와 용질이 균일하게 섞이게 위해서 초음파를 이용한 Sonication을 진행하였다.
\begin{figure}[H]
	\begin{center}
		\begin{tabular}{ccc}
			\includegraphics[width=0.3\textwidth]{sonicator}&
			\includegraphics[width=0.3\textwidth]{spin_coating}&
			\includegraphics[width=0.3\textwidth]{hotplate}
		\end{tabular}
	\end{center}
	\caption{From top to bottom, sonicating, spin coating, PDMS stamping on hot plate.}
	\label{fig:sample}  
\end{figure}


Silicon wafer 위에 spin coating을 이용하여 용액을 균일하게 펼쳐주었다. 2000rpm으로 1분간 회전시켜주었고 미리 100도로 달궈놓았던 핫플레이트에서 5분간 PDMS를 이용하여 눌러주었다.
\begin{figure}[H]
	\begin{center}
		\begin{tabular}{cc}
			\includegraphics[width=0.45\textwidth]{OM}
		\end{tabular}
	\end{center}
	\caption{A silicon wafer taken with an OM (optical microscope).}
	\label{fig:om}  
\end{figure}
PDMS stamping 과정을 거친 이후에 silicon wafer에 결정이 잘 형성되었는지 확인하기 위해 OM(광학현미경)으로 1차적인 확인을 해주었다. 그 결과 Figure \ref{fig:om} 에서 잘 형성된 결정 여럿을 관찰할 수 있었고, 그 중 가장 잘 형성된 하나의 결정을 통해 연구를 진행하였다.


\subsection{데이터 추출}
제작된 sample을 NT-MDT 기기를 통하여 PL mapping 하였다. 이 데이터는 레이저의 조리개를 $OD = 2$ 로 맞춰놓은 ND2 상태에서 측정하였다. 생성된 단결정에 측정할 위치를 정해 놓고 PL을 측정하였다.  PL mapping이란, sample의 각 위치에서의 PL 데이터를 모두 담은 파일을 만드는 과정이다. 이렇게 만들어진 파일에서는 임의의 점에서의 PL data 를 얻어낼 수 있다는 장점이 있다.다
\begin{figure}[H]
	\begin{center}
		\begin{tabular}{cc}
			\includegraphics[width=0.65\textwidth]{Nova_screen_capture}&
			\includegraphics[width=0.4\textwidth]{line123}
		\end{tabular}
	\end{center}
	\caption{Extracting data by using Nova-Px program.}
	\label{fig:nova}  
\end{figure}
Nova Px 프로그램을 활용하여 PL mapping 된 파일에서 데이터를 각 점 별로 뽑아내었다. Figure \ref{fig:nova}와 같은 화면에서 초록색 십자의 위치를 조절하여 원하는 위치의 PL peak을 얻어낼 수 있다. 중앙에서부터 바깥으로 나갈 때의 PL peak 의 경향성을 알아보기 위해 Figure \ref{fig:nova} 의 오른쪽 사진에서 볼 수 있는 1, 2, 3 경로의 데이터를 추출하였다.

중앙에서부터 바깥 쪽으로 나가는 경로에서의 PL data를 추출해낸다. 중앙으로 잡은 점의 좌표는 (59.0, 53.6, 33)이다. (이때 좌표를 (x, y, z) 라 했을 때 x, y는 사진상에서의 좌표, z는 그림에서 보이는 밝기의 크기, 즉 PL peak의 대략적인 상대적 크기이다.) 그림 상으로는 정중앙이 아닐 수 있지만 PL peak이 가장 높게 나온 곳이므로 올바른 경향성을 찾아내기 위하여 설정 하였다. 설정된 중앙으로부터 바깥 방향으로 나가는 line 1, 2, 3 를 다음과 같이 설정하였다.

\begin{table}[H]%[width=1.0\linewidth]
	\caption{Routing lines 1, 2, and 3}
	\label{table01}
	\centering
	\begin{tabular}{c c}
	\toprule
	경로 번호 & 경로\\
	\toprule
	Line 1 & (59.0, 53.6, 33)-->(62.3, 56.9, 14) / +(0.4, 0.4) 씩 8번, 점 9개\\
	Line 2 & (59.0, 53.6, 33)-->(68.0, 51.3, 13) / +(0.8, -0.2) 씩 11번, 점 12개\\
	Line 3 & (59.0, 53.6, 33)-->(64.7, 47.9, 17) / +(0.4, -0.4) 씩 14번, 점 15개\\
	\toprule
	\end{tabular}
	\end{table}

중앙으로 잡은 점을 point 0, 각 line에 대해 있는 점들을 point 1-1, 1-2, … , 1-8, 2-1, 2-1, … , 2-11, 3-1, 3-2, … , 3-14로 정의하도록 하자. 또한 line 1 은 point 0 부터 point 1-8, line 2 은 point 0 부터 point 2-11, line 3 은 point 0 부터 point 3-14 까지 이다. 

\subsection{분석 과정}
\subsubsection{point data peak fitting}
각 점들의 추출된 data를 분석하기 위해서는 Origin 9 프로그램을 사용하였다. Chen (2018) 에 의하면 CsPbBr3 에서 biexciton과 exciton peak이 나타나는 wave length 는 각각 약 580nm, 600nm 이다\cite{chen2018room}. 이 사실을 바탕으로 PL data에서 보여진 peak을 두개의 peak의 합으로 fitting 하였다. Peak fitting 을 할 때에 gauss 매커니즘을 사용하였으며, biexciton 과 exciton이 존재하는 wave length에 peak 위치를 설정한 후 fitting을 진행하였다. Figure \ref{fig:point0}은 그 중 하나의 예시이다.
\begin{figure}[H]
	\begin{center}
		\begin{tabular}{cc}
			\includegraphics[width=0.8\textwidth]{point0}
		\end{tabular}
	\end{center}
	\caption{The PL data of the set point 0 is shown by sum of exciton and biexciton peak.}
	\label{fig:point0}  
\end{figure}
다음과 같이 multiple peak fitting 을 마친 후에는 각 peak의 x값, 즉 wavelength 값과 y값, 즉 intensity 값을 데이터로 기록한 후 분석하였다.
\subsubsection{line data analyze}
위의 과정에서 각 point 들의 data 에 대한 peak fitting 을 한 이후에 그 경향성을 보기 위해 필요한 과정이다. 분석하고자 하는 것은 중앙에서 바깥으로 가면서 peak intensity의 경향성이다. 이를 위해서 peak fitting 과정에서 얻은 데이터인 각 point 에서의 biexciton, exciton peak 의 intensity 값을 y축, point 번호를 x 축으로 설정하여  line 1, line 2, line 3 별로 막대그래프를 그려서 경향성을 볼 수 있었다.
\subsection{분석 결과 및 해석}
Line 1, Line 2, Line 3 에서의 결과는 각각 Figure \ref{fig:line1}, Figure \ref{fig:line2}, Figure \ref{fig:line3} 와 같이 나타난다.
\begin{figure}[H]
	\begin{center}
		\begin{tabular}{cc}
			\includegraphics[width=1\textwidth]{_line1}
		\end{tabular}
	\end{center}
	\caption{line 1 data analysis}
	\label{fig:line1}  
\end{figure}
Figure \ref{fig:line1}, 즉 line 1에서는 exciton과 biexciton 모두 감소하는 추세를 보이다가 끝에서 증가하는 모습을 볼 수 있다.
\begin{figure}[H]
	\begin{center}
		\begin{tabular}{cc}
			\includegraphics[width=1\textwidth]{_line2}
		\end{tabular}
	\end{center}
	\caption{line 2 data analysis}
	\label{fig:line2}  
\end{figure}
Figure \ref{fig:line2}, 즉 line 2에서는 exciton과 biexciton 모두 감소하는 추세를 보이다가 가장 끝 두점에서는 biexciton은 급격히 증가, exciton은 급격히 감소함을 볼 수 있다.

\begin{figure}[H]
	\begin{center}
		\begin{tabular}{cc}
			\includegraphics[width=1\textwidth]{_line3}
		\end{tabular}
	\end{center}
	\caption{line 3 data analysis}
	\label{fig:line3}  
\end{figure}
Figure \ref{fig:line3} , 즉 line 3에서는 exciton은 감소, biexciton은 증가하는 추세를 보이다가 가장 끝 두점에서는 biexciton은 급격히 감소, exciton은 급격히 증가함을 볼 수 있다.

세 line에서 exciton, biexciton 각각의 공통되는 경향성이나 규칙은 찾아보기 어렵다. 하지만 중앙에서 중간까지 갈 때는 특정한 경향성을 보이는 듯 하다가 가장 바깥, 가장자리에서 그 경향성이 반대가 되는 모습을 볼 수 있다. 종합적으로 보았을 때에는 가장자리로 가면서 감소하는 모습을 보이다가 다시 증가하는 모습이 세 line모두에서 나타나게 된다.
