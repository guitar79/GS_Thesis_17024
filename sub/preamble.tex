\documentclass{gshs_thesis}
\graphicspath{{images/}}
% 이곳에 필요한 별도의 패키지들을 적어넣으시오.
%\usepackage{...}
\usepackage{verbatim} % for commment, verbatim environment
\usepackage{spverbatim} % automatic linebreak verbatim environment
%\usepacakge{indentfirst}
\usepackage{tikz}
%\tikzset{
%	image label/.style={
%		every node/.style={
			%fill=black,
			%text=white,
%			font=\sffamily\scriptsize,
%			anchor=south west,
%			xshift=0,
%			yshift=0,
%			at={(0,0)}
%		}
%	}
%}
\usepackage{amsmath}
\usepackage{amsfonts}
\usepackage{amssymb}
\usepackage{float}
\usepackage{graphicx}
\usepackage{tabularx}
\usepackage{multirow}
\usepackage{booktabs}
\usepackage{longtable}
\usepackage{gensymb}
%\usepackage{subcaption}
%\usepackage{floatrow}
%\usepackage{pict2e}

\usepackage{pgfplots}
\pgfplotsset{
	compat=newest,
	label style={font=\sffamily\scriptsize},
	ticklabel style={font=\sffamily\scriptsize},
	legend style={font=\sffamily\tiny},
	major tick length=0.1cm,
	minor tick length=0.05cm,
	every x tick/.style={black},
}

\usetikzlibrary{shapes}
\usetikzlibrary{plotmarks}
\usepackage{listings}
\usepackage{hologo}
\usepackage{makecell}


\lstset{
	basicstyle=\small\ttfamily,
	columns=flexible,
	breaklines=true
}

\citation
\bibdata



% -----------------------------------------------------------------------
%                   이 부분은 수정하지 마시오.
% -----------------------------------------------------------------------
\titleheader{졸업논문청구논문}
\school{과학영재학교 경기과학고등학교}
\approval{위 논문은 과학영재학교 경기과학고등학교 졸업논문으로\\
졸업논문심사위원회에서 심사 통과하였음.}
\chairperson{심사위원장}
\examiner{심사위원}
\apprvsign{(인)}
\korabstract{초 록}
\koracknowledgement{감사의 글}
\korresearches{연 구 활 동}

%: ----------------------------------------------------------------------
%:                  논문 제목과 저자 이름을 입력하시오
% ----------------------------------------------------------------------
\title{단결정 페로브스카이트의 위치에 따른 PL측정을 통한 waveguiding effect의 원인분석} %한글 제목
\engtitle{Analysis of waveguiding effect by PL measurement according to the position of single crystal perovskite} %영문 제목
\korname{김 주 원} %저자 이름을 한글로 입력하시오 (글자 사이 띄어쓰기)
\engname{Kim, Ju Won} %저자 이름을 영어로 입력하시오 (family name, personal name)
\chnname{金 宙 源} %저자 이름을 한자로 입력하시오 (글자 사이 띄어쓰기)
\studid{17024} %학번을 입력하시오

%------------------------------------------------------------------------
%                  심사위원과 논문 승인 날짜를 입력하시오
%------------------------------------------------------------------------
\advisor{Park, Kie Hyun}  %지도교사 영문 이름 (family name, personal name)
\judgeone{정 문 석} %심사위원장
\judgetwo{김 제 흥}   %심사위원1
\judgethree{박 기 현} %심사위원2(지도교사)
\degreeyear{2019}   %졸업 년도
\degreedate{2019}{7}{21} %논문 승인 날짜 양식

%------------------------------------------------------------------------
%                  논문제출 전 체크리스트를 확인하시오
%------------------------------------------------------------------------
\checklisttitle{[논문제출 전 체크리스트]} %수정하지 마시오
\checklistI{1. 이 논문은 내가 직접 연구하고 작성한 것이다.} %수정하지 마시오
% 이 항목이 사실이라면 다음 줄 앞에 "%"기호 삽입, 다다음 줄 앞의 "%"기호 제거하시오
%\checklistmarkI{$\square$}
\checklistmarkI{$\text{\rlap{$\checkmark$}}\square$}
\checklistII{2. 인용한 모든 자료(책, 논문, 인터넷자료 등)의 인용표시를 바르게 하였다.} %수정하지 마시오
% 이 항목이 사실이라면 다음 줄 앞에 "%"기호 삽입, 다다음 줄 앞의 "%"기호 제거하시오
%\checklistmarkII{$\square$}
\checklistmarkII{$\text{\rlap{$\checkmark$}}\square$}
\checklistIII{3. 인용한 자료의 표현이나 내용을 왜곡하지 않았다.} %수정하지마시오
% 이 항목이 사실이라면 다음 줄 앞에 "%"기호 삽입, 다다음 줄 앞의 "%"기호 제거하시오
%\checklistmarkIII{$\square$}
\checklistmarkIII{$\text{\rlap{$\checkmark$}}\square$}
\checklistIV{4. 정확한 출처제시 없이 다른 사람의 글이나 아이디어를 가져오지 않았다.} %수정하지 마시오
% 이 항목이 사실이라면 다음 줄 앞에 "%"기호 삽입, 다다음 줄 앞의 "%"기호 제거하시오
%\checklistmarkIV{$\square$}
\checklistmarkIV{$\text{\rlap{$\checkmark$}}\square$}
\checklistV{5. 논문 작성 중 도표나 데이터를 조작(위조 혹은 변조)하지 않았다.} %수정하지 마시오
% 이 항목이 사실이라면 다음 줄 앞에 "%"기호 삽입, 다다음 줄 앞의 "%"기호 제거하시오
%\checklistmarkV{$\square$}
\checklistmarkV{$\text{\rlap{$\checkmark$}}\square$}
\checklistVI{6. 다른 친구와 같은 내용의 논문을 제출하지 않았다.} %수정하지 마시오
% 이 항목이 사실이라면 다음 줄 앞에 "%"기호 삽입, 다다음 줄 앞의 "%"기호 제거하시오
%\checklistmarkVI{$\square$}
\checklistmarkVI{$\text{\rlap{$\checkmark$}}\square$}